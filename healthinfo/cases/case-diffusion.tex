\documentclass[]{article}
\usepackage{lmodern}
\usepackage{amssymb,amsmath}
\usepackage{ifxetex,ifluatex}
\usepackage{fixltx2e} % provides \textsubscript
\ifnum 0\ifxetex 1\fi\ifluatex 1\fi=0 % if pdftex
  \usepackage[T1]{fontenc}
  \usepackage[utf8]{inputenc}
\else % if luatex or xelatex
  \ifxetex
    \usepackage{mathspec}
  \else
    \usepackage{fontspec}
  \fi
  \defaultfontfeatures{Ligatures=TeX,Scale=MatchLowercase}
\fi
% use upquote if available, for straight quotes in verbatim environments
\IfFileExists{upquote.sty}{\usepackage{upquote}}{}
% use microtype if available
\IfFileExists{microtype.sty}{%
\usepackage{microtype}
\UseMicrotypeSet[protrusion]{basicmath} % disable protrusion for tt fonts
}{}
\usepackage[margin=1in]{geometry}
\usepackage{hyperref}
\hypersetup{unicode=true,
            pdftitle={妗堜緥锛氫腑蹇冩€т笌鏂颁骇鍝佹墿鏁},
            pdfauthor={鍚寸繑},
            pdfborder={0 0 0},
            breaklinks=true}
\urlstyle{same}  % don't use monospace font for urls
\usepackage{color}
\usepackage{fancyvrb}
\newcommand{\VerbBar}{|}
\newcommand{\VERB}{\Verb[commandchars=\\\{\}]}
\DefineVerbatimEnvironment{Highlighting}{Verbatim}{commandchars=\\\{\}}
% Add ',fontsize=\small' for more characters per line
\usepackage{framed}
\definecolor{shadecolor}{RGB}{248,248,248}
\newenvironment{Shaded}{\begin{snugshade}}{\end{snugshade}}
\newcommand{\AlertTok}[1]{\textcolor[rgb]{0.94,0.16,0.16}{#1}}
\newcommand{\AnnotationTok}[1]{\textcolor[rgb]{0.56,0.35,0.01}{\textbf{\textit{#1}}}}
\newcommand{\AttributeTok}[1]{\textcolor[rgb]{0.77,0.63,0.00}{#1}}
\newcommand{\BaseNTok}[1]{\textcolor[rgb]{0.00,0.00,0.81}{#1}}
\newcommand{\BuiltInTok}[1]{#1}
\newcommand{\CharTok}[1]{\textcolor[rgb]{0.31,0.60,0.02}{#1}}
\newcommand{\CommentTok}[1]{\textcolor[rgb]{0.56,0.35,0.01}{\textit{#1}}}
\newcommand{\CommentVarTok}[1]{\textcolor[rgb]{0.56,0.35,0.01}{\textbf{\textit{#1}}}}
\newcommand{\ConstantTok}[1]{\textcolor[rgb]{0.00,0.00,0.00}{#1}}
\newcommand{\ControlFlowTok}[1]{\textcolor[rgb]{0.13,0.29,0.53}{\textbf{#1}}}
\newcommand{\DataTypeTok}[1]{\textcolor[rgb]{0.13,0.29,0.53}{#1}}
\newcommand{\DecValTok}[1]{\textcolor[rgb]{0.00,0.00,0.81}{#1}}
\newcommand{\DocumentationTok}[1]{\textcolor[rgb]{0.56,0.35,0.01}{\textbf{\textit{#1}}}}
\newcommand{\ErrorTok}[1]{\textcolor[rgb]{0.64,0.00,0.00}{\textbf{#1}}}
\newcommand{\ExtensionTok}[1]{#1}
\newcommand{\FloatTok}[1]{\textcolor[rgb]{0.00,0.00,0.81}{#1}}
\newcommand{\FunctionTok}[1]{\textcolor[rgb]{0.00,0.00,0.00}{#1}}
\newcommand{\ImportTok}[1]{#1}
\newcommand{\InformationTok}[1]{\textcolor[rgb]{0.56,0.35,0.01}{\textbf{\textit{#1}}}}
\newcommand{\KeywordTok}[1]{\textcolor[rgb]{0.13,0.29,0.53}{\textbf{#1}}}
\newcommand{\NormalTok}[1]{#1}
\newcommand{\OperatorTok}[1]{\textcolor[rgb]{0.81,0.36,0.00}{\textbf{#1}}}
\newcommand{\OtherTok}[1]{\textcolor[rgb]{0.56,0.35,0.01}{#1}}
\newcommand{\PreprocessorTok}[1]{\textcolor[rgb]{0.56,0.35,0.01}{\textit{#1}}}
\newcommand{\RegionMarkerTok}[1]{#1}
\newcommand{\SpecialCharTok}[1]{\textcolor[rgb]{0.00,0.00,0.00}{#1}}
\newcommand{\SpecialStringTok}[1]{\textcolor[rgb]{0.31,0.60,0.02}{#1}}
\newcommand{\StringTok}[1]{\textcolor[rgb]{0.31,0.60,0.02}{#1}}
\newcommand{\VariableTok}[1]{\textcolor[rgb]{0.00,0.00,0.00}{#1}}
\newcommand{\VerbatimStringTok}[1]{\textcolor[rgb]{0.31,0.60,0.02}{#1}}
\newcommand{\WarningTok}[1]{\textcolor[rgb]{0.56,0.35,0.01}{\textbf{\textit{#1}}}}
\usepackage{graphicx,grffile}
\makeatletter
\def\maxwidth{\ifdim\Gin@nat@width>\linewidth\linewidth\else\Gin@nat@width\fi}
\def\maxheight{\ifdim\Gin@nat@height>\textheight\textheight\else\Gin@nat@height\fi}
\makeatother
% Scale images if necessary, so that they will not overflow the page
% margins by default, and it is still possible to overwrite the defaults
% using explicit options in \includegraphics[width, height, ...]{}
\setkeys{Gin}{width=\maxwidth,height=\maxheight,keepaspectratio}
\IfFileExists{parskip.sty}{%
\usepackage{parskip}
}{% else
\setlength{\parindent}{0pt}
\setlength{\parskip}{6pt plus 2pt minus 1pt}
}
\setlength{\emergencystretch}{3em}  % prevent overfull lines
\providecommand{\tightlist}{%
  \setlength{\itemsep}{0pt}\setlength{\parskip}{0pt}}
\setcounter{secnumdepth}{0}
% Redefines (sub)paragraphs to behave more like sections
\ifx\paragraph\undefined\else
\let\oldparagraph\paragraph
\renewcommand{\paragraph}[1]{\oldparagraph{#1}\mbox{}}
\fi
\ifx\subparagraph\undefined\else
\let\oldsubparagraph\subparagraph
\renewcommand{\subparagraph}[1]{\oldsubparagraph{#1}\mbox{}}
\fi

%%% Use protect on footnotes to avoid problems with footnotes in titles
\let\rmarkdownfootnote\footnote%
\def\footnote{\protect\rmarkdownfootnote}

%%% Change title format to be more compact
\usepackage{titling}

% Create subtitle command for use in maketitle
\newcommand{\subtitle}[1]{
  \posttitle{
    \begin{center}\large#1\end{center}
    }
}

\setlength{\droptitle}{-2em}

  \title{妗堜緥锛氫腑蹇冩€т笌鏂颁骇鍝佹墿鏁}
    \pretitle{\vspace{\droptitle}\centering\huge}
  \posttitle{\par}
    \author{鍚寸繑}
    \preauthor{\centering\large\emph}
  \postauthor{\par}
      \predate{\centering\large\emph}
  \postdate{\par}
    \date{2018-11-04}


\begin{document}
\maketitle

\subsection{概述}

我们通过案例来阐述如何使用社会网络分析研究健康管理。所有分析过程均通过R语言实现。

本案例源自市场营销中的常用场景:

\begin{quote}
新产品在初期扩散较为缓慢的时候,是否可能借助意见领袖的口碑来促进产品的扩散?
\end{quote}

接下来,我们将通过论文导读与R语言实现两个部分了解中心性在市场营销中的具体应用。

\subsection{问题背景}

2012年谷歌眼镜的亮相,被称作``智能穿戴设备元年''。在智能手机的创新空间逐步收窄和市场增量接近饱和的情况下,智能穿戴设备作为智能终端产业下一个热点已被市场广泛认同。

智能穿戴设备可以用作:

\begin{itemize}
\tightlist
\item
  慢性病管理
\item
  生活与运动习惯监测
\end{itemize}

而今智能手环、智能手表等产品已经非常丰富,然而大多数产品的销售状况并不理想。在市场营销中,为了促进新产品的扩散,企业可能采用``product
seeding program'',亦即:

\begin{quote}
选取意见领袖作为``种子客户'',免费向其提供新产品,以期这些``种子客户''能够通过口碑效应,促进新产品的扩散。
\end{quote}

然而,社会网络中的哪些客户是``意见领袖''呢?

我们求诸节点中心性。

\subsection{研究思路}

对于任意行动者\(i\),购买智能穿戴设备的决策受到两个途径的影响:

\begin{itemize}
\tightlist
\item
  大众传播:例如广告等途径,其特点是具有全局效果(可以近似认为广告覆盖了所有目标人群)
\item
  口碑传播:通过个体之间传播,其特点是具有局部效果(每个人只受其家人朋友的影响)
\end{itemize}

因此,如果第\(t - 1\)期末行动者\(i\)尚未购买智能穿戴设备,那么他在第\(i\)期购买智能穿戴设备的概率是:

\begin{equation}
  \text{prob}_{it} = 1 - (1 - p) \times (1 - q) ^ {m_{it}},
\end{equation}

其中:

\begin{itemize}
\tightlist
\item
  \(p\):创新系数,用以刻画大众传播
\item
  \(q\):模仿系数,用以刻画口碑传播
\item
  \(m_{it}\):在第\(t - 1\)期末,与行动者\(i\)直接相连的行动者中,已经购买了智能穿戴设备的数量
\end{itemize}

如果行动者\(i\)在第\(t\)期购买了智能穿戴设备,则记作\(y_{it} = 1\);否则,记作记作\(y_{it} = 0\)。所以,每期的销售量为:

\begin{equation}
  \text{sales}_{t} = \sum_{i} y_{it}.
\end{equation}

进一步,考虑到货币的时间价值,可以通过折现率\(r\)来计算企业通过销售智能穿戴设备获得的收益净现值:

\begin{equation}
  \text{npv} = \sum_{t} \frac{\text{sales}_{t}}{(1 + r)^{t}}.
\end{equation}

\subsection{正常产品扩散情形}

我们首先搭建仿真模型,以模拟该新型智能穿戴设备在投放市场之后的扩散过程。

\subsubsection{创建并初始化社会网络}

\begin{Shaded}
\begin{Highlighting}[]
\KeywordTok{rm}\NormalTok{(}\DataTypeTok{list =} \KeywordTok{ls}\NormalTok{())}
\KeywordTok{suppressMessages}\NormalTok{(}\KeywordTok{library}\NormalTok{(igraph))}
\KeywordTok{suppressMessages}\NormalTok{(}\KeywordTok{library}\NormalTok{(ggplot2))}
\KeywordTok{suppressMessages}\NormalTok{(}\KeywordTok{library}\NormalTok{(dplyr))}
\KeywordTok{suppressMessages}\NormalTok{(}\KeywordTok{library}\NormalTok{(tidyr))}
\KeywordTok{set.seed}\NormalTok{(}\DecValTok{123}\NormalTok{)}

\CommentTok{# create a graph}
\NormalTok{n <-}\StringTok{ }\DecValTok{5000}
\NormalTok{g <-}\StringTok{ }\KeywordTok{random.graph.game}\NormalTok{(}\DataTypeTok{n =}\NormalTok{ n, }\DataTypeTok{p =} \FloatTok{0.002}\NormalTok{)}

\CommentTok{# set parameters}
\NormalTok{setattr <-}\StringTok{ }\ControlFlowTok{function}\NormalTok{(g, }\DataTypeTok{seeds =} \OtherTok{NULL}\NormalTok{) \{}
    \CommentTok{# set vertex attributes}
    \KeywordTok{V}\NormalTok{(g)}\OperatorTok{$}\NormalTok{adopted <-}\StringTok{ }\DecValTok{0}
    \ControlFlowTok{if}\NormalTok{ (}\KeywordTok{length}\NormalTok{(seeds) }\OperatorTok{>}\StringTok{ }\DecValTok{0}\NormalTok{) \{}
      \CommentTok{# seeds}
      \KeywordTok{V}\NormalTok{(g)}\OperatorTok{$}\NormalTok{adopted[seeds] <-}\StringTok{ }\DecValTok{1}
\NormalTok{    \}}
    \CommentTok{# set graph attributes}
    \KeywordTok{graph_attr}\NormalTok{(g, }\StringTok{"adopters"}\NormalTok{) <-}\StringTok{ }\KeywordTok{list}\NormalTok{(}\KeywordTok{sum}\NormalTok{(}\KeywordTok{V}\NormalTok{(g)}\OperatorTok{$}\NormalTok{adopted))}
    \CommentTok{# return the graph}
    \KeywordTok{return}\NormalTok{(g)}
\NormalTok{\}}
\end{Highlighting}
\end{Shaded}

\subsubsection{新产品扩散机制}

智能穿戴设备的扩散过程中:

\begin{itemize}
\tightlist
\item
  行动者之间是相互沟通和影响的
\item
  影响模式由以上公式刻画
\end{itemize}

\begin{Shaded}
\begin{Highlighting}[]
\NormalTok{iteration <-}\StringTok{ }\ControlFlowTok{function}\NormalTok{(g, }\DataTypeTok{p =} \FloatTok{0.1}\NormalTok{, }\DataTypeTok{q =} \FloatTok{0.01}\NormalTok{) \{}
    \CommentTok{# Number of vertices}
\NormalTok{    N <-}\StringTok{ }\KeywordTok{vcount}\NormalTok{(g)}
    \CommentTok{# find non-adopters}
\NormalTok{    non_adopters <-}\StringTok{ }\KeywordTok{as.vector}\NormalTok{(}\KeywordTok{V}\NormalTok{(g)[}\KeywordTok{V}\NormalTok{(g)}\OperatorTok{$}\NormalTok{adopted }\OperatorTok{==}\StringTok{ }\DecValTok{0}\NormalTok{])}
    \CommentTok{# get the adjacency matrix}
\NormalTok{    adjacencymat <-}\StringTok{ }\KeywordTok{as_adjacency_matrix}\NormalTok{(g, }\DataTypeTok{sparse =} \OtherTok{TRUE}\NormalTok{)}
    \CommentTok{# extract relevant adjacency matrix}
\NormalTok{    na_adjacency <-}\StringTok{ }\NormalTok{adjacencymat[non_adopters, , drop =}\StringTok{ }\OtherTok{FALSE}\NormalTok{]}
    \CommentTok{# remove the large matrix object}
    \KeywordTok{rm}\NormalTok{(adjacencymat)}
    \CommentTok{# initialize the number of adopters}
\NormalTok{    nums_adopted <-}\StringTok{ }\KeywordTok{unlist}\NormalTok{(g}\OperatorTok{$}\NormalTok{adopters)}
    \CommentTok{# initialize the number of periods in which continuous zero adopttion occurs}
\NormalTok{    num_test <-}\StringTok{ }\DecValTok{0}
    \CommentTok{# iterate over all none-adopted vertices}
    \ControlFlowTok{while}\NormalTok{ (}\KeywordTok{length}\NormalTok{(non_adopters) }\OperatorTok{!=}\StringTok{ }\DecValTok{0} \OperatorTok{&&}\StringTok{ }\NormalTok{num_test }\OperatorTok{<}\StringTok{ }\DecValTok{10}\NormalTok{) \{}
        \CommentTok{# adopted judge}
        \CommentTok{# number of i's acquaintances who have adopted the good by time t}
\NormalTok{        mit <-}\StringTok{ }\KeywordTok{as.vector}\NormalTok{(na_adjacency }\OperatorTok\StringTok{ }\KeywordTok{V}\NormalTok{(g)}\OperatorTok{$}\NormalTok{adopted)}
\NormalTok{        pit <-}\StringTok{ }\DecValTok{1} \OperatorTok{-}\StringTok{ }\NormalTok{(}\DecValTok{1} \OperatorTok{-}\StringTok{ }\NormalTok{p) }\OperatorTok{*}\StringTok{ }\NormalTok{(}\DecValTok{1} \OperatorTok{-}\StringTok{ }\NormalTok{q) }\OperatorTok{^}\StringTok{ }\NormalTok{mit}
\NormalTok{        flag <-}\StringTok{ }\KeywordTok{runif}\NormalTok{(}\KeywordTok{length}\NormalTok{(non_adopters)) }\OperatorTok{<}\StringTok{ }\NormalTok{pit}
        \CommentTok{# update the adopted status}
\NormalTok{        flagid <-}\StringTok{ }\NormalTok{non_adopters[}\KeywordTok{which}\NormalTok{(flag }\OperatorTok{==}\StringTok{ }\OtherTok{TRUE}\NormalTok{)]}
        \ControlFlowTok{if}\NormalTok{ (}\KeywordTok{length}\NormalTok{(flagid)) \{}
            \KeywordTok{V}\NormalTok{(g)[flagid]}\OperatorTok{$}\NormalTok{adopted <-}\StringTok{ }\OtherTok{TRUE}
\NormalTok{        \}}
        \CommentTok{# update the non-adopters id}
\NormalTok{        nonflagid <-}\StringTok{ }\KeywordTok{which}\NormalTok{(flag }\OperatorTok{==}\StringTok{ }\OtherTok{FALSE}\NormalTok{)}
\NormalTok{        non_adopters <-}\StringTok{ }\NormalTok{non_adopters[flag }\OperatorTok{==}\StringTok{ }\OtherTok{FALSE}\NormalTok{]}
        \CommentTok{# update the relevant adjacency matrix}
\NormalTok{        na_adjacency <-}\StringTok{ }\NormalTok{na_adjacency[nonflagid, , drop =}\StringTok{ }\OtherTok{FALSE}\NormalTok{]}
        \CommentTok{# update the number of new adopters}
\NormalTok{        nums_adopted <-}\StringTok{ }\KeywordTok{c}\NormalTok{(nums_adopted, }\KeywordTok{sum}\NormalTok{(flag))}
        \CommentTok{# update the number of periods in which continuous zero adopttion occurs}
        \ControlFlowTok{if}\NormalTok{ (}\KeywordTok{sum}\NormalTok{(flag) }\OperatorTok{==}\StringTok{ }\DecValTok{0}\NormalTok{) \{}
\NormalTok{            num_test <-}\StringTok{ }\NormalTok{num_test }\OperatorTok{+}\StringTok{ }\DecValTok{1}
\NormalTok{        \}}\ControlFlowTok{else}\NormalTok{ \{}
\NormalTok{            num_test <-}\StringTok{ }\DecValTok{0}
\NormalTok{        \}}
\NormalTok{    \}}
    \CommentTok{# store as a graph attribute}
    \KeywordTok{graph_attr}\NormalTok{(g, }\StringTok{"adopters"}\NormalTok{) <-}\StringTok{ }\KeywordTok{list}\NormalTok{(nums_adopted)}
    \CommentTok{# return the graph}
    \KeywordTok{return}\NormalTok{(g)}
\NormalTok{\}}
\end{Highlighting}
\end{Shaded}

根据新产品扩散过程,我们可以计算对应的净现值(net present value, NPV)。

\begin{Shaded}
\begin{Highlighting}[]
\CommentTok{# create a function to calculate the NPV}
\NormalTok{npv <-}\StringTok{ }\ControlFlowTok{function}\NormalTok{(sales, }\DataTypeTok{discount.rate =} \FloatTok{0.05}\NormalTok{) \{}
  \CommentTok{# calculate npv}
\NormalTok{  npv <-}\StringTok{ }\KeywordTok{sum}\NormalTok{(sales }\OperatorTok{/}\StringTok{ }\NormalTok{(}\DecValTok{1} \OperatorTok{+}\StringTok{ }\NormalTok{discount.rate) }\OperatorTok{^}\StringTok{ }\NormalTok{(}\DecValTok{1}\OperatorTok{:}\KeywordTok{length}\NormalTok{(sales)))}
  \KeywordTok{return}\NormalTok{(npv)}
\NormalTok{\}}
\end{Highlighting}
\end{Shaded}

\subsubsection{新产品销售量}

我们根据以上仿真模型,计算该智能穿戴设备的销售量。

\begin{Shaded}
\begin{Highlighting}[]
\CommentTok{# base model}
\CommentTok{# set attributes}
\NormalTok{g.base <-}\StringTok{ }\KeywordTok{setattr}\NormalTok{(g)}
\CommentTok{# diffusion process}
\NormalTok{g.base <-}\StringTok{ }\KeywordTok{iteration}\NormalTok{(g.base)}
\CommentTok{# obtain the sales}
\NormalTok{sales.base <-}\StringTok{ }\KeywordTok{graph_attr}\NormalTok{(g.base, }\StringTok{"adopters"}\NormalTok{)[[}\DecValTok{1}\NormalTok{]]}
\NormalTok{sales.base <-}\StringTok{ }\KeywordTok{data.frame}\NormalTok{(}\DataTypeTok{t =} \DecValTok{1}\OperatorTok{:}\KeywordTok{length}\NormalTok{(sales.base), }\DataTypeTok{basesales =}\NormalTok{ sales.base)}
\CommentTok{# plot the sales}
\KeywordTok{ggplot}\NormalTok{(sales.base, }\KeywordTok{aes}\NormalTok{(}\DataTypeTok{x =}\NormalTok{ t, }\DataTypeTok{y =}\NormalTok{ basesales)) }\OperatorTok{+}\StringTok{ }\KeywordTok{geom_line}\NormalTok{() }\OperatorTok{+}\StringTok{ }\KeywordTok{labs}\NormalTok{(}\DataTypeTok{caption =} \StringTok{"Sales over periods for the new wearable device"}\NormalTok{) }\OperatorTok{+}\StringTok{ }\KeywordTok{theme_bw}\NormalTok{()}
\end{Highlighting}
\end{Shaded}

\includegraphics{case-diffusion_files/figure-latex/unnamed-chunk-4-1.pdf}

同时可以看到,总销售量为4999。

\subsubsection{净现值}

考虑到货币的时间价值(2017年的1000元比2018年的1000元人民币更值钱:通货膨胀),我们计算销售额的净现值。

\begin{Shaded}
\begin{Highlighting}[]
\CommentTok{# print the results}
\NormalTok{npv.base <-}\StringTok{ }\KeywordTok{npv}\NormalTok{(sales.base}\OperatorTok{$}\NormalTok{basesales)}
\NormalTok{npvvec <-}\StringTok{ }\KeywordTok{data.frame}\NormalTok{(}\DataTypeTok{npv.base =}\NormalTok{ npv.base)}
\NormalTok{npvvec}
\end{Highlighting}
\end{Shaded}

\begin{verbatim}
##   npv.base
## 1     3443
\end{verbatim}

\subsection{营销策略及结果}

我们依次考虑四种营销策略。

\subsubsection{度中心性}

首先,我们计算节点的度中心性,并绘制度中心性的分布图。

\begin{Shaded}
\begin{Highlighting}[]
\CommentTok{# calculate degree centrality}
\NormalTok{dc <-}\StringTok{ }\KeywordTok{degree}\NormalTok{(g)}
\CommentTok{# distribution of dc}
\KeywordTok{data.frame}\NormalTok{(}\DataTypeTok{dc =}\NormalTok{ dc) }\OperatorTok\StringTok{ }\KeywordTok{ggplot}\NormalTok{(}\KeywordTok{aes}\NormalTok{(}\DataTypeTok{x =}\NormalTok{ dc)) }\OperatorTok{+}\StringTok{ }\KeywordTok{geom_histogram}\NormalTok{(}\DataTypeTok{binwidth =} \FloatTok{0.5}\NormalTok{) }\OperatorTok{+}\StringTok{ }\KeywordTok{theme_bw}\NormalTok{()}
\end{Highlighting}
\end{Shaded}

\includegraphics{case-diffusion_files/figure-latex/unnamed-chunk-6-1.pdf}

进而,我们选取10个度中心性最大的行动者(即消费者)作为``种子客户'',免费向其提供智能穿戴产品。

\begin{Shaded}
\begin{Highlighting}[]
\CommentTok{# select 2/1000 actors as seeds}
\NormalTok{seedsize <-}\StringTok{ }\NormalTok{n }\OperatorTok{*}\StringTok{ }\FloatTok{0.002}
\NormalTok{seeds.dc <-}\StringTok{ }\KeywordTok{order}\NormalTok{(dc, }\DataTypeTok{decreasing =}\NormalTok{ T)[}\DecValTok{1}\OperatorTok{:}\NormalTok{seedsize]}
\CommentTok{# print the dc of these seeds}
\NormalTok{dc[seeds.dc]}
\end{Highlighting}
\end{Shaded}

\begin{verbatim}
##  [1] 23 22 22 21 21 21 21 21 20 20
\end{verbatim}

可以看到,以上``种子客户''的度(未归一化)都大约在20左右。

\begin{Shaded}
\begin{Highlighting}[]
\CommentTok{# set attributes}
\NormalTok{g.dc <-}\StringTok{ }\KeywordTok{setattr}\NormalTok{(g, }\DataTypeTok{seeds =}\NormalTok{ seeds.dc)}
\CommentTok{# diffusion process}
\NormalTok{g.dc <-}\StringTok{ }\KeywordTok{iteration}\NormalTok{(g.dc)}
\CommentTok{# obtain the sales}
\NormalTok{sales.dc <-}\StringTok{ }\KeywordTok{graph_attr}\NormalTok{(g.dc, }\StringTok{"adopters"}\NormalTok{)[[}\DecValTok{1}\NormalTok{]]}

\CommentTok{# create a data frame}
\NormalTok{t <-}\StringTok{ }\DecValTok{1}\OperatorTok{:}\KeywordTok{max}\NormalTok{(}\KeywordTok{length}\NormalTok{(sales.base}\OperatorTok{$}\NormalTok{basesales), }\KeywordTok{length}\NormalTok{(sales.dc))}
\CommentTok{# adding zeros}
\NormalTok{addzeros <-}\StringTok{ }\ControlFlowTok{function}\NormalTok{(x, tmax) \{}
  \ControlFlowTok{if}\NormalTok{ (}\KeywordTok{length}\NormalTok{(x) }\OperatorTok{<}\StringTok{ }\NormalTok{tmax) \{}
    \CommentTok{# adding zeros}
\NormalTok{    x <-}\StringTok{ }\KeywordTok{c}\NormalTok{(x, }\KeywordTok{rep}\NormalTok{(}\DecValTok{0}\NormalTok{, tmax }\OperatorTok{-}\StringTok{ }\KeywordTok{length}\NormalTok{(x)))}
\NormalTok{  \}}
  \CommentTok{# return}
  \KeywordTok{return}\NormalTok{(x)}
\NormalTok{\}}
\CommentTok{# apply for two vectors}
\NormalTok{illust.dat <-}\StringTok{ }\KeywordTok{data.frame}\NormalTok{(t, }\DataTypeTok{normal =} \KeywordTok{addzeros}\NormalTok{(sales.base}\OperatorTok{$}\NormalTok{basesales, }\KeywordTok{length}\NormalTok{(t)), }\DataTypeTok{dc =} \KeywordTok{addzeros}\NormalTok{(sales.dc, }\KeywordTok{length}\NormalTok{(t)))}
\NormalTok{illust.dat <-}\StringTok{ }\KeywordTok{gather}\NormalTok{(illust.dat, }\StringTok{"type"}\NormalTok{, }\StringTok{"adopters"}\NormalTok{, }\DecValTok{2}\OperatorTok{:}\DecValTok{3}\NormalTok{)}
\CommentTok{# plot the results}
\KeywordTok{ggplot}\NormalTok{(illust.dat, }\KeywordTok{aes}\NormalTok{(}\DataTypeTok{x =}\NormalTok{ t, }\DataTypeTok{y =}\NormalTok{ adopters, }\DataTypeTok{linetype =}\NormalTok{ type)) }\OperatorTok{+}\StringTok{ }\KeywordTok{geom_line}\NormalTok{() }\OperatorTok{+}\StringTok{ }\KeywordTok{labs}\NormalTok{(}\DataTypeTok{caption =} \StringTok{"Sales over periods for the new wearable device"}\NormalTok{) }\OperatorTok{+}\StringTok{ }\KeywordTok{theme_bw}\NormalTok{()}
\end{Highlighting}
\end{Shaded}

\includegraphics{case-diffusion_files/figure-latex/unnamed-chunk-8-1.pdf}

可以看到,总销售量为5000。

\begin{Shaded}
\begin{Highlighting}[]
\CommentTok{# calculate and compare NPV}
\NormalTok{npv.dc <-}\StringTok{ }\KeywordTok{npv}\NormalTok{(sales.dc)}
\NormalTok{npvvec <-}\StringTok{ }\KeywordTok{cbind}\NormalTok{(npvvec, npv.dc)}
\NormalTok{npvvec}
\end{Highlighting}
\end{Shaded}

\begin{verbatim}
##   npv.base npv.dc
## 1     3443   3446
\end{verbatim}

\subsubsection{特征向量中心性}

首先,我们计算节点的特征向量中心性,并绘制特征向量中心性的分布图。

\begin{Shaded}
\begin{Highlighting}[]
\CommentTok{# calculate eigenvector centrality}
\NormalTok{ec <-}\StringTok{ }\KeywordTok{eigen_centrality}\NormalTok{(g, }\DataTypeTok{scale =}\NormalTok{ T)}\OperatorTok{$}\NormalTok{vector}
\CommentTok{# distribution of ec}
\KeywordTok{data.frame}\NormalTok{(}\DataTypeTok{ec =}\NormalTok{ ec) }\OperatorTok\StringTok{ }\KeywordTok{ggplot}\NormalTok{(}\KeywordTok{aes}\NormalTok{(}\DataTypeTok{x =}\NormalTok{ ec)) }\OperatorTok{+}\StringTok{ }\KeywordTok{geom_histogram}\NormalTok{(}\DataTypeTok{binwidth =} \FloatTok{5e-3}\NormalTok{) }\OperatorTok{+}\StringTok{ }\KeywordTok{theme_bw}\NormalTok{()}
\end{Highlighting}
\end{Shaded}

\includegraphics{case-diffusion_files/figure-latex/unnamed-chunk-10-1.pdf}

进而,我们选取10个特征向量中心性最大的行动者(即消费者)作为``种子客户'',免费向其提供智能穿戴产品。

\begin{Shaded}
\begin{Highlighting}[]
\CommentTok{# select 2/1000 actors as seeds}
\NormalTok{seeds.ec <-}\StringTok{ }\KeywordTok{order}\NormalTok{(ec, }\DataTypeTok{decreasing =}\NormalTok{ T)[}\DecValTok{1}\OperatorTok{:}\NormalTok{seedsize]}
\CommentTok{# print the cc of these seeds}
\NormalTok{ec[seeds.ec]}
\end{Highlighting}
\end{Shaded}

\begin{verbatim}
##  [1] 1.000 0.926 0.915 0.889 0.878 0.872 0.866 0.862 0.861 0.845
\end{verbatim}

\begin{Shaded}
\begin{Highlighting}[]
\CommentTok{# set attributes}
\NormalTok{g.ec <-}\StringTok{ }\KeywordTok{setattr}\NormalTok{(g, }\DataTypeTok{seeds =}\NormalTok{ seeds.ec)}
\CommentTok{# diffusion process}
\NormalTok{g.ec <-}\StringTok{ }\KeywordTok{iteration}\NormalTok{(g.ec)}
\CommentTok{# obtain the sales}
\NormalTok{sales.ec <-}\StringTok{ }\KeywordTok{graph_attr}\NormalTok{(g.ec, }\StringTok{"adopters"}\NormalTok{)[[}\DecValTok{1}\NormalTok{]]}

\CommentTok{# create a data frame}
\NormalTok{t <-}\StringTok{ }\DecValTok{1}\OperatorTok{:}\KeywordTok{max}\NormalTok{(}\KeywordTok{length}\NormalTok{(sales.base}\OperatorTok{$}\NormalTok{basesales), }\KeywordTok{length}\NormalTok{(sales.dc), }\KeywordTok{length}\NormalTok{(sales.ec))}
\CommentTok{# adding zeros}
\NormalTok{addzeros <-}\StringTok{ }\ControlFlowTok{function}\NormalTok{(x, tmax) \{}
  \ControlFlowTok{if}\NormalTok{ (}\KeywordTok{length}\NormalTok{(x) }\OperatorTok{<}\StringTok{ }\NormalTok{tmax) \{}
    \CommentTok{# adding zeros}
\NormalTok{    x <-}\StringTok{ }\KeywordTok{c}\NormalTok{(x, }\KeywordTok{rep}\NormalTok{(}\DecValTok{0}\NormalTok{, tmax }\OperatorTok{-}\StringTok{ }\KeywordTok{length}\NormalTok{(x)))}
\NormalTok{  \}}
  \CommentTok{# return}
  \KeywordTok{return}\NormalTok{(x)}
\NormalTok{\}}
\CommentTok{# apply for two vectors}
\NormalTok{illust.dat <-}\StringTok{ }\KeywordTok{data.frame}\NormalTok{(t,}
                         \DataTypeTok{normal =} \KeywordTok{addzeros}\NormalTok{(sales.base}\OperatorTok{$}\NormalTok{basesales, }\KeywordTok{length}\NormalTok{(t)),}
                         \DataTypeTok{dc =} \KeywordTok{addzeros}\NormalTok{(sales.dc, }\KeywordTok{length}\NormalTok{(t)),}
                         \DataTypeTok{ec =} \KeywordTok{addzeros}\NormalTok{(sales.ec, }\KeywordTok{length}\NormalTok{(t)))}
\NormalTok{illust.dat <-}\StringTok{ }\KeywordTok{gather}\NormalTok{(illust.dat, }\StringTok{"type"}\NormalTok{, }\StringTok{"adopters"}\NormalTok{, }\DecValTok{2}\OperatorTok{:}\DecValTok{4}\NormalTok{)}
\CommentTok{# plot the results}
\KeywordTok{ggplot}\NormalTok{(illust.dat, }\KeywordTok{aes}\NormalTok{(}\DataTypeTok{x =}\NormalTok{ t, }\DataTypeTok{y =}\NormalTok{ adopters, }\DataTypeTok{linetype =}\NormalTok{ type)) }\OperatorTok{+}\StringTok{ }\KeywordTok{geom_line}\NormalTok{() }\OperatorTok{+}\StringTok{ }\KeywordTok{labs}\NormalTok{(}\DataTypeTok{caption =} \StringTok{"Sales over periods for the new wearable device"}\NormalTok{) }\OperatorTok{+}\StringTok{ }\KeywordTok{theme_bw}\NormalTok{()}
\end{Highlighting}
\end{Shaded}

\includegraphics{case-diffusion_files/figure-latex/unnamed-chunk-12-1.pdf}

可以看到,总销售量为5000。

\begin{Shaded}
\begin{Highlighting}[]
\CommentTok{# calculate and compare NPV}
\NormalTok{npv.ec <-}\StringTok{ }\KeywordTok{npv}\NormalTok{(sales.ec)}
\NormalTok{npvvec <-}\StringTok{ }\KeywordTok{cbind}\NormalTok{(npvvec, npv.ec)}
\NormalTok{npvvec}
\end{Highlighting}
\end{Shaded}

\begin{verbatim}
##   npv.base npv.dc npv.ec
## 1     3443   3446   3442
\end{verbatim}

\subsubsection{接近中心性}

首先,我们计算节点的接近中心性,并绘制接近中心性的分布图。

\begin{Shaded}
\begin{Highlighting}[]
\CommentTok{# calculate closeness centrality}
\NormalTok{cc <-}\StringTok{ }\KeywordTok{closeness}\NormalTok{(g, }\DataTypeTok{normalized =}\NormalTok{ T)}
\CommentTok{# distribution of cc}
\KeywordTok{data.frame}\NormalTok{(}\DataTypeTok{cc =}\NormalTok{ cc) }\OperatorTok\StringTok{ }\KeywordTok{ggplot}\NormalTok{(}\KeywordTok{aes}\NormalTok{(}\DataTypeTok{x =}\NormalTok{ cc)) }\OperatorTok{+}\StringTok{ }\KeywordTok{geom_histogram}\NormalTok{(}\DataTypeTok{binwidth =} \FloatTok{5e-4}\NormalTok{) }\OperatorTok{+}\StringTok{ }\KeywordTok{theme_bw}\NormalTok{()}
\end{Highlighting}
\end{Shaded}

\includegraphics{case-diffusion_files/figure-latex/unnamed-chunk-14-1.pdf}

进而,我们选取10个接近中心性最大的行动者(即消费者)作为``种子客户'',免费向其提供智能穿戴产品。

\begin{Shaded}
\begin{Highlighting}[]
\CommentTok{# select 2/1000 actors as seeds}
\NormalTok{seeds.cc <-}\StringTok{ }\KeywordTok{order}\NormalTok{(cc, }\DataTypeTok{decreasing =}\NormalTok{ T)[}\DecValTok{1}\OperatorTok{:}\NormalTok{seedsize]}
\CommentTok{# print the cc of these seeds}
\NormalTok{cc[seeds.cc]}
\end{Highlighting}
\end{Shaded}

\begin{verbatim}
##  [1] 0.280 0.279 0.278 0.278 0.278 0.278 0.277 0.277 0.277 0.277
\end{verbatim}

\begin{Shaded}
\begin{Highlighting}[]
\CommentTok{# set attributes}
\NormalTok{g.cc <-}\StringTok{ }\KeywordTok{setattr}\NormalTok{(g, }\DataTypeTok{seeds =}\NormalTok{ seeds.cc)}
\CommentTok{# diffusion process}
\NormalTok{g.cc <-}\StringTok{ }\KeywordTok{iteration}\NormalTok{(g.cc)}
\CommentTok{# obtain the sales}
\NormalTok{sales.cc <-}\StringTok{ }\KeywordTok{graph_attr}\NormalTok{(g.cc, }\StringTok{"adopters"}\NormalTok{)[[}\DecValTok{1}\NormalTok{]]}

\CommentTok{# create a data frame}
\NormalTok{t <-}\StringTok{ }\DecValTok{1}\OperatorTok{:}\KeywordTok{max}\NormalTok{(}\KeywordTok{length}\NormalTok{(sales.base}\OperatorTok{$}\NormalTok{basesales), }\KeywordTok{length}\NormalTok{(sales.dc), }\KeywordTok{length}\NormalTok{(sales.ec), }\KeywordTok{length}\NormalTok{(sales.cc))}
\CommentTok{# adding zeros}
\NormalTok{addzeros <-}\StringTok{ }\ControlFlowTok{function}\NormalTok{(x, tmax) \{}
  \ControlFlowTok{if}\NormalTok{ (}\KeywordTok{length}\NormalTok{(x) }\OperatorTok{<}\StringTok{ }\NormalTok{tmax) \{}
    \CommentTok{# adding zeros}
\NormalTok{    x <-}\StringTok{ }\KeywordTok{c}\NormalTok{(x, }\KeywordTok{rep}\NormalTok{(}\DecValTok{0}\NormalTok{, tmax }\OperatorTok{-}\StringTok{ }\KeywordTok{length}\NormalTok{(x)))}
\NormalTok{  \}}
  \CommentTok{# return}
  \KeywordTok{return}\NormalTok{(x)}
\NormalTok{\}}
\CommentTok{# apply for two vectors}
\NormalTok{illust.dat <-}\StringTok{ }\KeywordTok{data.frame}\NormalTok{(t,}
                         \DataTypeTok{normal =} \KeywordTok{addzeros}\NormalTok{(sales.base}\OperatorTok{$}\NormalTok{basesales, }\KeywordTok{length}\NormalTok{(t)),}
                         \DataTypeTok{dc =} \KeywordTok{addzeros}\NormalTok{(sales.dc, }\KeywordTok{length}\NormalTok{(t)),}
                         \DataTypeTok{ec =} \KeywordTok{addzeros}\NormalTok{(sales.ec, }\KeywordTok{length}\NormalTok{(t)),}
                         \DataTypeTok{cc =} \KeywordTok{addzeros}\NormalTok{(sales.cc, }\KeywordTok{length}\NormalTok{(t)))}
\NormalTok{illust.dat <-}\StringTok{ }\KeywordTok{gather}\NormalTok{(illust.dat, }\StringTok{"type"}\NormalTok{, }\StringTok{"adopters"}\NormalTok{, }\DecValTok{2}\OperatorTok{:}\DecValTok{5}\NormalTok{)}
\CommentTok{# plot the results}
\KeywordTok{ggplot}\NormalTok{(illust.dat, }\KeywordTok{aes}\NormalTok{(}\DataTypeTok{x =}\NormalTok{ t, }\DataTypeTok{y =}\NormalTok{ adopters, }\DataTypeTok{linetype =}\NormalTok{ type)) }\OperatorTok{+}\StringTok{ }\KeywordTok{geom_line}\NormalTok{() }\OperatorTok{+}\StringTok{ }\KeywordTok{labs}\NormalTok{(}\DataTypeTok{caption =} \StringTok{"Sales over periods for the new wearable device"}\NormalTok{) }\OperatorTok{+}\StringTok{ }\KeywordTok{theme_bw}\NormalTok{()}
\end{Highlighting}
\end{Shaded}

\includegraphics{case-diffusion_files/figure-latex/unnamed-chunk-16-1.pdf}

可以看到,总销售量为4999。

\begin{Shaded}
\begin{Highlighting}[]
\CommentTok{# calculate and compare NPV}
\NormalTok{npv.cc <-}\StringTok{ }\KeywordTok{npv}\NormalTok{(sales.cc)}
\NormalTok{npvvec <-}\StringTok{ }\KeywordTok{cbind}\NormalTok{(npvvec, npv.cc)}
\NormalTok{npvvec}
\end{Highlighting}
\end{Shaded}

\begin{verbatim}
##   npv.base npv.dc npv.ec npv.cc
## 1     3443   3446   3442   3417
\end{verbatim}

\subsubsection{中介中心性}

首先,我们计算节点的中介中心性,并绘制中介中心性的分布图。

\begin{Shaded}
\begin{Highlighting}[]
\CommentTok{# calculate betweenness centrality}
\NormalTok{bc <-}\StringTok{ }\KeywordTok{betweenness}\NormalTok{(g, }\DataTypeTok{normalized =}\NormalTok{ T)}
\CommentTok{# distribution of bc}
\KeywordTok{data.frame}\NormalTok{(}\DataTypeTok{bc =}\NormalTok{ bc) }\OperatorTok\StringTok{ }\KeywordTok{ggplot}\NormalTok{(}\KeywordTok{aes}\NormalTok{(}\DataTypeTok{x =}\NormalTok{ bc)) }\OperatorTok{+}\StringTok{ }\KeywordTok{geom_histogram}\NormalTok{(}\DataTypeTok{binwidth =} \FloatTok{5e-5}\NormalTok{) }\OperatorTok{+}\StringTok{ }\KeywordTok{theme_bw}\NormalTok{()}
\end{Highlighting}
\end{Shaded}

\includegraphics{case-diffusion_files/figure-latex/unnamed-chunk-18-1.pdf}

进而,我们选取10个中介中心性最大的行动者(即消费者)作为``种子客户'',免费向其提供智能穿戴产品。

\begin{Shaded}
\begin{Highlighting}[]
\CommentTok{# select 2/1000 actors as seeds}
\NormalTok{seeds.bc <-}\StringTok{ }\KeywordTok{order}\NormalTok{(bc, }\DataTypeTok{decreasing =}\NormalTok{ T)[}\DecValTok{1}\OperatorTok{:}\NormalTok{seedsize]}
\CommentTok{# print the bc of these seeds}
\NormalTok{bc[seeds.bc]}
\end{Highlighting}
\end{Shaded}

\begin{verbatim}
##  [1] 0.00279 0.00250 0.00242 0.00231 0.00231 0.00230 0.00229 0.00229
##  [9] 0.00227 0.00224
\end{verbatim}

\begin{Shaded}
\begin{Highlighting}[]
\CommentTok{# set attributes}
\NormalTok{g.bc <-}\StringTok{ }\KeywordTok{setattr}\NormalTok{(g, }\DataTypeTok{seeds =}\NormalTok{ seeds.bc)}
\CommentTok{# diffusion process}
\NormalTok{g.bc <-}\StringTok{ }\KeywordTok{iteration}\NormalTok{(g.bc)}
\CommentTok{# obtain the sales}
\NormalTok{sales.bc <-}\StringTok{ }\KeywordTok{graph_attr}\NormalTok{(g.bc, }\StringTok{"adopters"}\NormalTok{)[[}\DecValTok{1}\NormalTok{]]}

\CommentTok{# create a data frame}
\NormalTok{t <-}\StringTok{ }\DecValTok{1}\OperatorTok{:}\KeywordTok{max}\NormalTok{(}\KeywordTok{length}\NormalTok{(sales.base}\OperatorTok{$}\NormalTok{basesales), }\KeywordTok{length}\NormalTok{(sales.dc), }\KeywordTok{length}\NormalTok{(sales.ec), }\KeywordTok{length}\NormalTok{(sales.cc), }\KeywordTok{length}\NormalTok{(sales.bc))}
\CommentTok{# adding zeros}
\NormalTok{addzeros <-}\StringTok{ }\ControlFlowTok{function}\NormalTok{(x, tmax) \{}
  \ControlFlowTok{if}\NormalTok{ (}\KeywordTok{length}\NormalTok{(x) }\OperatorTok{<}\StringTok{ }\NormalTok{tmax) \{}
    \CommentTok{# adding zeros}
\NormalTok{    x <-}\StringTok{ }\KeywordTok{c}\NormalTok{(x, }\KeywordTok{rep}\NormalTok{(}\DecValTok{0}\NormalTok{, tmax }\OperatorTok{-}\StringTok{ }\KeywordTok{length}\NormalTok{(x)))}
\NormalTok{  \}}
  \CommentTok{# return}
  \KeywordTok{return}\NormalTok{(x)}
\NormalTok{\}}
\CommentTok{# apply for two vectors}
\NormalTok{illust.dat <-}\StringTok{ }\KeywordTok{data.frame}\NormalTok{(t,}
                         \DataTypeTok{normal =} \KeywordTok{addzeros}\NormalTok{(sales.base}\OperatorTok{$}\NormalTok{basesales, }\KeywordTok{length}\NormalTok{(t)),}
                         \DataTypeTok{dc =} \KeywordTok{addzeros}\NormalTok{(sales.dc, }\KeywordTok{length}\NormalTok{(t)),}
                         \DataTypeTok{ec =} \KeywordTok{addzeros}\NormalTok{(sales.ec, }\KeywordTok{length}\NormalTok{(t)),}
                         \DataTypeTok{cc =} \KeywordTok{addzeros}\NormalTok{(sales.cc, }\KeywordTok{length}\NormalTok{(t)),}
                         \DataTypeTok{bc =} \KeywordTok{addzeros}\NormalTok{(sales.bc, }\KeywordTok{length}\NormalTok{(t)))}
\NormalTok{illust.dat <-}\StringTok{ }\KeywordTok{gather}\NormalTok{(illust.dat, }\StringTok{"type"}\NormalTok{, }\StringTok{"adopters"}\NormalTok{, }\DecValTok{2}\OperatorTok{:}\DecValTok{6}\NormalTok{)}
\CommentTok{# plot the results}
\KeywordTok{ggplot}\NormalTok{(illust.dat, }\KeywordTok{aes}\NormalTok{(}\DataTypeTok{x =}\NormalTok{ t, }\DataTypeTok{y =}\NormalTok{ adopters, }\DataTypeTok{linetype =}\NormalTok{ type)) }\OperatorTok{+}\StringTok{ }\KeywordTok{geom_line}\NormalTok{() }\OperatorTok{+}\StringTok{ }\KeywordTok{labs}\NormalTok{(}\DataTypeTok{caption =} \StringTok{"Sales over periods for the new wearable device"}\NormalTok{) }\OperatorTok{+}\StringTok{ }\KeywordTok{theme_bw}\NormalTok{()}
\end{Highlighting}
\end{Shaded}

\includegraphics{case-diffusion_files/figure-latex/unnamed-chunk-20-1.pdf}

可以看到,总销售量为5000。

最后,我们可以比较四种策略下的销售额净现值。

\begin{Shaded}
\begin{Highlighting}[]
\CommentTok{# calculate and compare NPV}
\NormalTok{npv.bc <-}\StringTok{ }\KeywordTok{npv}\NormalTok{(sales.bc)}
\NormalTok{npvvec <-}\StringTok{ }\KeywordTok{cbind}\NormalTok{(npvvec, npv.bc)}
\NormalTok{npvvec}
\end{Highlighting}
\end{Shaded}

\begin{verbatim}
##   npv.base npv.dc npv.ec npv.cc npv.bc
## 1     3443   3446   3442   3417   3460
\end{verbatim}


\end{document}
